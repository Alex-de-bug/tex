\documentclass{article}
\usepackage[utf8]{inputenc} %кодировка
\usepackage[T2A]{fontenc}
\usepackage[english,russian]{babel} %русификатор 
\usepackage{mathtools} %библиотека матеши
\usepackage[left=1cm,right=1cm,top=2cm,bottom=2cm,bindingoffset=0cm]{geometry} %изменение отступов на листе
\usepackage{amsmath}
\usepackage{graphicx} %библиотека для графики и картинок
\graphicspath{}
\DeclareGraphicsExtensions{.pdf,.png,.jpg}
\usepackage{subcaption}
\usepackage{pgfplots}
\usepackage{float}
\usepackage{graphicx}
\usepackage{hyperref}

\begin{document}

\tableofcontents
\newpage

\section[Перепись]{Перепись}
\subsection[Санта-клаус]{С помощью ИС «Санта-клаус» детям дарят подарки за хорошее поведение. Разработать модель требований, Use-Case модель и доменную модель.}
\subsubsection{Модель требований FRUPS+}
Функциональные требования
\begin{itemize}
    \item SEC0 - Система должна обеспечивать регистрацию пользователей с помощью имени пользователя и пароля.
    \item SEC1 - Система должна поддерживать аутентификацию пользователей с помощью имени пользователя и пароля или подтверждения с помощью СМС или email письма.
    \item FR0 - Система должна поддерживать добавления, удаления, редактирование информации (не качества детей) о детях.
    \item FR1 - Система должна обеспечивать возможноть отмечать, удалять поведенческие достижения их детей.
    \item FR2 - Система должна поддреживать просмотр каталога подарков.
    \item FR3 - Система должна поддреживать возможность составления списка интересных подарков для ребёнка.
    \item FR4 - Система должна предоставлять возможность восстановления забытого пароля.
    \item FR5 - Система должна поддреживать редактирование и мониторинг каталога подарков для админов.
\end{itemize}
Нефункциональные требования
\begin{itemize}
    \item USA0 - Система должна обеспечивать адаптивный дизайн для различных устройств.
    \item RELI0 - Система должна быть доступна 99.9\% времени, с автоматическим восстановлением после сбоев.
    \item PERF0 - Система должна обрабатывать тысячи запросов одновременно без существенных задержек.
    \item SUPP0 - Система должна легко масштабироваться для поддержки увеличения числа пользователей.
\end{itemize}
\subsubsection{Use-case диаграмма}
\begin{center}
    \includegraphics[width=.7\textwidth]{use-case.png}
\end{center}
\subsubsection{Доменная модель}
Основные сущности:

\begin{itemize}
    \item Пользователь (атрибуты: имя, роль, контактная информация, аутентификационные данные)
    \item Ребенок (атрибуты: имя, возраст, баллы поведения)
    \item Подарок (атрибуты: наименование, количество баллов для заказа, запасы)
    \item Заказ (атрибуты: дата, статус, связь с ребенком и выбранным подарком)
\end{itemize}








\section[Термины]{Термины}
\subsection[Модель требований FRUPS+]{Модель требований FRUPS+}
Подробное описание того что должно быть реализовано системой, но при этом не должно описывать, как его нужно реализовать.
Включает в себя функциональные требования, нефункциональные. 

Функциональные определяют, что система должна делать: наборы функциональных требований (FR + номер), возможности ПО (СAP + номер), требования к безопасности (SEC + номер).
Наборы - набор свойств продукты необходимый для выполнения конкретной деятельности (сис должна обеспечивать ввод, модификацию и удаление данных о клиенте).
Требования к безопасности - метод аунтификаци, список ролей, шифрование, хранение данных в защищённых источниках (сис должна обеспечивать двухфакторную аутентификацию пользователей с помощью имени пользователя и пароля и подтверждения с помощью СМС.)

Нефункциональные: 

Usaility - учёт особенностей пользователя (быстрота ответа в интервале), эстетические требавания (цвет, расстояния между элементами), согласованность пользовательсткого интерфейса, согласованность пользовательского интерфейса, требования к справочной подсистеме, требования к пользовательской документации, требования к учебным материалам.

Reliability - частота и обработка заказов, способность системы восстанавливать продуктивное функционирование, предсказуемость поведения, точность, среднее время между отказами. Требования к надёжности, которые предназначены для способности ПО безотказно функционировать. 
В требованиях обычно указывается допустимое число отказов и сбоев за определённый промежуток времени. Способность системы восстанавливать продуктивное функционирование в течение заданного времени.
Требованием является accuracy - точность, например, проведения вычислений. Коэффициент готовности системы — отношение времени исправной работы к сумме времён исправной работы и вынужденных простоев объекта, взятых за один и тот же календарный срок.

Performance - скорость решения задач, эффективность, готовность системы к решению задач, пропускная способность, время отклика, время восстановления, использование системных ресурсов.
Требованием является скорость решения вычислительных задач. Также скорость важна в длительных инженерных расчетах, когда необходимо выполнить, например, моделирование за разумное для человека время.
Требования к эффективности фиксируют процент времени, которое тратится на выполнение полезных задач, по отношению к времени на выполнение общесистемных.
Требованием к производительности является готовность (availability) быстро начать выполнение задачи.
Какой объём данных или запросов система может обработать за единицу времени.
Для большой реактивности придется пожертвовать пропускной способностью.

Supportability - способность системы к расширению и масштабированию и выполнению большего объема обработки данных. Адаптируемость под конкретные задачи, поддерживаемость.
Требования к совместимости позволяют использовать различные операционные системы, версии продуктов, браузеров и пр. совместно с разработанным ПО. Отдельно выделяются системные требования и минимальные требования к установке системы, например, объём
ОЗУ, количество и частота процессоров и пр.

\subsection[Доменная модель]{Доменная модель}
Доменная модель — это концептуальная модель предметной области, которая отображает ключевые сущности, их атрибуты и взаимосвязи между ними, а также основные правила бизнес-логики. Эта модель помогает разработчикам и всем участникам проекта лучше понять структуру и функционирование системы, на которой они работают.


\end{document}
\includegraphics[width=.9\textwidth]{123}