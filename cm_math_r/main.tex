\documentclass{article}
\usepackage[utf8]{inputenc} %кодировка
\usepackage[T2A]{fontenc}
\usepackage[english,russian]{babel} %русификатор 
\usepackage{mathtools} %библиотека матеши
\usepackage[left=1cm,right=1cm,top=2cm,bottom=2cm,bindingoffset=0cm]{geometry} %изменение отступов на листе
\usepackage{amsmath}
\usepackage{graphicx} %библиотека для графики и картинок
\graphicspath{}
\DeclareGraphicsExtensions{.pdf,.png,.jpg}
\usepackage{subcaption}
\usepackage{pgfplots}
\usepackage{float}

\begin{document}
\section{интегралы}
$h = \frac{b-a}{n}$\\
Прямоугольники\\
$I_r = h\sum_{i=1}^{n}y_i$\\
$I_c = h\sum_{i=1}^{n}y_{i-\frac{1}{2}}$\\
$I_l = h\sum_{i=1}^{n}y_{i-1}$
\\ \\
Трапеции\\
$I = h(\frac{y_0+y_n}{2}+\sum_{i=1}^{n-1}y_i)$
\\ \\
Симпсон\\
$I = \frac{n}{3}(y_0+4\cdot(y_1+y_3+...+y_{n-1})+ \\+2\cdot(y_2+y_4+...+y_{n-2})+y_n)$

\section{Линейные системы}
Гаусс-Зейдель\\
$\begin{cases}
    2x_1+2x_2+10x_3 = 14\\
    10x_1+x_2+x_3 = 12\\
    2x_1+10x_2+x_3=13
\end{cases}$
$\begin{cases}
    x_1 = -0.1x_2 -0.1x_3+1.2\\
    x_2 = -0.2x_1 -0.1x_3+1.3\\
    x_3 = -0.2x_1 -0.2x_2+1.4\\
\end{cases}$
\\ 
C = 
$\begin{pmatrix}
    0&-0.1&-0.1\\
    -0.2&0&-0.1\\
    -0.2&-0.2&0\\
\end{pmatrix}$; 
d = 
$\begin{pmatrix}
    1.2\\
    1.3\\ 
    1.4
\end{pmatrix}$
||C|| = 0.4 < 1;
\\
$max|x_i^{(k)}-x_i^{(k-1)}|<\varepsilon$
\\
$x_1^{k+1} = -0.1x_2^k-0.1x_3^k+1.2$\\
$x_2^{k+1} = -0.2x_1^{k+1}-0.1x_3^k+1.3$,  $x^0 = d$\\
$x_3^{k+1} = -0.2x_1^{k+1}-0.2x_2^{k+1}+1.4$\\
Простой итерации\\
$x_1^{k+1} = -0.1x_2^k-0.1x_3^k+1.2$\\
$x_2^{k+1} = -0.2x_1^k-0.1x_3^k+1.3$,  $x^0 = d$\\
$x_3^{k+1} = -0.2x_1^k-0.2x_2^k+1.4$\\
\section{Нелинейные системы}
Половинного деления\\
$x = \frac{a+b}{2}$; \\
$f(a)\cdot f(x)>0$ => $b=x$; $f(a)\cdot f(x)<=0$ => $a=x$\\
$n=n+1$; оценка - $|a-b|<=\varepsilon$ или $|f(x)|<\varepsilon$\\
Если конец, то $x = \frac{a+b}{2}$
\\ \\
Ньютона\\


\end{document}
