\documentclass{article}
\usepackage[utf8]{inputenc} %кодировка
\usepackage[T2A]{fontenc}
\usepackage[english,russian]{babel} %русификатор 
\usepackage{mathtools} %библиотека матеши
\usepackage[left=1cm,right=1cm,top=2cm,bottom=2cm,bindingoffset=0cm]{geometry} %изменение отступов на листе
\usepackage{amsmath}
\usepackage{graphicx} %библиотека для графики и картинок
\graphicspath{}
\DeclareGraphicsExtensions{.pdf,.png,.jpg}
\usepackage{subcaption}
\usepackage{pgfplots}
\usepackage{float}

\begin{document}
\begin{tikzpicture}
  \begin{axis}[
    xlabel={Координата $X$, см},
    ylabel={Потенциал $\phi$, В},
    xmin=0, xmax=30,
    ymin=0, ymax=15,
    axis lines=middle,
    grid=both,
    grid style={line width=.1pt, draw=gray!10},
    major grid style={line width=.2pt,draw=gray!50},
    width=12cm, height=8cm,
    legend style={at={(0.5,-0.15)},anchor=north}
  ]

  % Точки данных
  \addplot [mark=*, only marks] coordinates {(3.5,2) (7.2,4) (10.5,6) (11,7.9) (16,7.9) (17,8) (21.2,10) (25.9,12)};
  
  % Линии между точками
  \addplot [smooth, thick, blue] coordinates {(3.5,2) (7.2,4) (10.5,6) (11,7.9)};
  \addplot [thick, blue] coordinates {(11,7.9) (16,7.9)};
  \addplot [smooth, thick, blue] coordinates {(11,7.9) (17,8) (21.2,10) (25.9,12)};
  
  \end{axis}
\end{tikzpicture}
\end{document}