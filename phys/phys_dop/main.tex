\documentclass{article}
\usepackage[utf8]{inputenc} %кодировка
\usepackage[T2A]{fontenc}
\usepackage[english,russian]{babel} %русификатор 
\usepackage{mathtools} %библиотека матеши
\usepackage[left=1cm,right=1cm,top=2cm,bottom=2cm,bindingoffset=0cm]{geometry} %изменение отступов на листе
\usepackage{amsmath}
\usepackage{graphicx} %библиотека для графики и картинок
\graphicspath{}
\DeclareGraphicsExtensions{.pdf,.png,.jpg}
\usepackage{subcaption}
\usepackage{pgfplots}
\usepackage{tikz}


\begin{document}

\begin{tikzpicture}
    \begin{axis}[
        xlabel={Интервалы},
        ylabel={Плотность вероятности},
        ymin=0, ymax=1.5,
        xmin=4.3, xmax=5.7,
        ytick={0.10, 0.2, 0.3, 0.4, 0.5, 0.6, 0.7, 0.8, 0.9, 1, 1.1, 1.2, 1.3, 1.4},
        xticklabel style={
            /pgf/number format/fixed,
            /pgf/number format/precision=5
        },
        scaled x ticks=false,
    ]
    
    % Добавляем столбцы гистограммы
    \addplot+[ybar interval, mark=no] plot coordinates {
        (4.3, 0.07) (4.49, 0.07) % Интервал от 4.30 до 4.49
        (4.5, 0.29) (4.69, 0.29) % Интервал от 4.50 до 4.69
        (4.7, 0.78) (4.89, 0.78) % Интервал от 4.70 до 4.89
        (4.9, 1.3) (5.09, 1.3)   % Интервал от 4.90 до 5.09
        (5.1, 1.33) (5.29, 1.33) % Интервал от 5.10 до 5.29
        (5.3, 0.84) (5.49, 0.84) % Интервал от 5.30 до 5.49
        (5.5, 0.33) (5.69, 0.33) % Интервал от 5.50 до 5.69
    };
    
    % Добавляем кривую плотности вероятности
    \addplot [smooth, thick, red] coordinates {
        (4.395, 0.07)
        (4.595, 0.29)
        (4.795, 0.78)
        (4.995, 1.3)
        (5.195, 1.33)
        (5.395, 0.84)
        (5.595, 0.33)
    };
    
    \end{axis}
    \end{tikzpicture}


\end{document}
