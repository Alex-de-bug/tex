\documentclass{article}
\usepackage[utf8]{inputenc} %кодировка
\usepackage[T2A]{fontenc}
\usepackage[english,russian]{babel} %русификатор 
\usepackage{mathtools} %библиотека матеши
\usepackage[left=1cm,right=1cm,top=2cm,bottom=2cm,bindingoffset=0cm]{geometry} %изменение отступов на листе
\usepackage{amsmath}
\usepackage{graphicx} %библиотека для графики и картинок
\graphicspath{}
\DeclareGraphicsExtensions{.pdf,.png,.jpg}
\usepackage{subcaption}
\usepackage{pgfplots}
\usepackage{float}
\usepackage{multirow}
\usepackage{listings}
\usepackage{xcolor}
\usepackage{tikz}
\usetikzlibrary{automata,positioning}



\lstset{
    backgroundcolor=\color{white},   % Цвет фона
    basicstyle=\footnotesize\ttfamily, % Шрифт
    breaklines=true,                  % Перенос строк
    frame=single,                     % Рамка вокруг кода
}


\begin{document}
% НАЧАЛО ТИТУЛЬНОГО ЛИСТА
\begin{center}
    \Large
    Федеральное государственное автономное \\
    образовательное учреждение высшего образования \\ 
    «Научно-образовательная корпорация ИТМО»\\
    \vspace{0.5cm}
    \large
    Факультет программной инженерии и компьютерной техники \\
    Направление подготовки 09.03.04 Программная инженерия \\
    \vspace{1cm}
    \Large
    \textbf{Отчёт по лабораторной работе № 2} \\
    По дисциплине «Моделирование» (семестр 5)\\
    \large
    \vspace{8cm}

    \begin{minipage}{.33\textwidth}
    \end{minipage}
    \hfill
    \begin{minipage}{.4\textwidth}
    
        \textbf{Студенты}: \vspace{.1cm} \\
        \ Дениченко Александр P3312\\
        \ Балин Артём P3312\\
        \ Кобелев Роман P3312\\
        \textbf{Практик}:  \\
        \ Мартынчук Илья Геннадьевич
    \end{minipage}
    \vfill
Санкт-Петербург\\ 2024 г.
\end{center}
\pagestyle{empty}
% КОНЕЦ ТИТУЛЬНОГО ЛИСТА 
\newpage
\pagestyle{plain}

\section*{Цель работы}
Изучение метода марковских случайных процессов и его применение для
исследования простейших моделей – систем массового обслуживания (СМО) с
однородным потоком заявок.

\section{Исходные данные}
Система 1: 

- Кол-во приборов: 2

- Ёмкость накопителей: 2/1 (для первого/второго приборов)
\\ \\
Система 2: 

- Кол-во приборов: 1

- Закон распределения длительности обслуживания: гиперэкспоненциальный с коэф-том вариации v = 1.5

- Ёмкость накопителя: 2
\\ \\
Критерий эффективности: минимальные потери заявок
\\ \\
Параметры загрузки (12 группа):

- интенсивнось потока: $\lambda = 0.7 \ c^{-1}$

- средняя длительность обслуживания: $b = 5\ c$

- вероятность занятия первого прибора: $p_1 = 0.8$

- вероятность занятия второго прибора: $p_2 = 0.2$
\\ \\
Интенсивность обслуживания: $\mu = \frac{1}{b} = 0.2 \ c^{-1}$
\\ \\
\section{Выполнение}
\subsection*{Состояния системы 1}

\begin{table}[h]
    \centering
    \begin{tabular}{|c|c|c|}
    \hline
    Комбинация & Обозначение & Значение \\
    \hline
    0/0/0/0 & S1 &  0.013593\\
    1/0/0/0 & S2 &  0.038061\\
    1/0/1/0 & S3 &  0.106570\\
    1/0/2/0 & S4 &  0.298397\\
    0/1/0/0 & S5 &  0.009515\\
    0/1/0/1 & S6 &  0.006661\\
    1/1/0/0 & S7 &  0.026643\\
    1/1/1/0 & S8 &  0.074599\\
    1/1/2/0 & S9 &  0.208878\\
    1/1/0/1 & S10 &  0.018650\\
    1/1/1/1 & S11 &  0.052219\\
    1/1/2/1 & S12 &  0.146214\\
    \hline
    \end{tabular}
    \caption{Система 1}
    \label{tab:system1}
\end{table}
Результат: Сумма вероятностей: 1.000000
\subsection*{Граф переходов системы 1}
\begin{center}
    \begin{tikzpicture}[
        ->,
        >=stealth,
        node distance=3cm,
        every state/.style={thick, fill=white},
        transform shape  % Масштабирует также текст и узлы
    ]
    \node[state] (S1) {$S_1$};
    \node[state] (S2) [right of=S1] {$S_2$};
    \node[state] (S3) [right of=S2] {$S_3$};
    \node[state] (S4) [right of=S3] {$S_4$};
    \node[state] (S5) [below of=S1] {$S_5$};
    \node[state] (S7) [right of=S5] {$S_7$};
    \node[state] (S8) [right of=S7] {$S_8$};
    \node[state] (S9) [right of=S8] {$S_9$};
    \node[state] (S6) [below of=S5] {$S_6$};
    \node[state] (S10) [right of=S6] {$S_{10}$};
    \node[state] (S11) [right of=S10] {$S_{11}$};
    \node[state] (S12) [right of=S11] {$S_{12}$};

        % 1
        \path (S1) edge[bend left] node[above] {$p_1\lambda$} (S2)
            (S2) edge[bend left] node[above] {$\mu$} (S1);

        \path (S5) edge[bend left] node[right] {$\mu$} (S1)
            (S1) edge[bend left] node[right] {$p_2\lambda$} (S5);
        
        % 2
        \path (S2) edge[bend left] node[above] {$p_1\lambda$} (S3)
            (S3) edge[bend left] node[above] {$\mu$} (S2);

        \path (S2) edge[bend left] node[right] {$p_2\lambda$} (S7)
            (S7) edge[bend left] node[right] {$\mu$} (S2);

        % 3
        \path (S3) edge[bend left] node[above] {$p_1\lambda$} (S4)
            (S4) edge[bend left] node[above] {$\mu$} (S3);

        \path (S3) edge[bend left] node[right] {$p_2\lambda$} (S8)
            (S8) edge[bend left] node[right] {$\mu$} (S3);

        % 4
        \path (S4) edge[bend left] node[right] {$p_2\lambda$} (S9)
            (S9) edge[bend left] node[right] {$\mu$} (S4);

        % 5
        \path (S5) edge[bend left] node[below] {$p_1\lambda$} (S7)
            (S7) edge[bend left] node[below] {$\mu$} (S5);

        \path (S5) edge[bend left] node[left] {$p_2\lambda$} (S6)
            (S6) edge[bend left] node[left] {$\mu$} (S5);

        % 6
        \path (S6) edge[bend left] node[below] {$p_1\lambda$} (S10)
            (S10) edge[bend left] node[below] {$\mu$} (S6);

        % 7
        \path (S7) edge[bend left] node[below] {$p_1\lambda$} (S8)
            (S8) edge[bend left] node[below] {$\mu$} (S7);

        \path (S7) edge[bend left] node[left] {$p_2\lambda$} (S10)
            (S10) edge[bend left] node[left] {$\mu$} (S7); 
        
        % 8
        \path (S8) edge[bend left] node[below] {$p_1\lambda$} (S9)
            (S9) edge[bend left] node[below] {$\mu$} (S8);

        \path (S8) edge[bend left] node[left] {$p_2\lambda$} (S11)
            (S11) edge[bend left] node[left] {$\mu$} (S8); 

        % 9
        \path (S9) edge[bend left] node[left] {$p_2\lambda$} (S12)
            (S12) edge[bend left] node[left] {$\mu$} (S9);

        % 10
        \path (S10) edge[bend left] node[below] {$p_1\lambda$} (S11)
            (S11) edge[bend left] node[below] {$\mu$} (S10);

        % 11
        \path (S11) edge[bend left] node[below] {$p_1\lambda$} (S12)
            (S12) edge[bend left] node[below] {$\mu$} (S11);


    \end{tikzpicture}
\end{center}
\subsection*{Матрица интенсивностей переходов системы 1}

\begin{table}[H]
    \centering
    \begin{tabular}{c|rrrrrrrrrrrr}
        & $S_1$ & $S_2$ & $S_3$ & $S_4$ & $S_5$ & $S_6$ & $S_7$ & $S_8$ & $S_9$ & $S_{10}$ & $S_{11}$ & $S_{12}$ \\
        \hline
        $S_1$     & -0.70 & 0.56 & 0.00 & 0.00 & 0.14 & 0.00 & 0.00 & 0.00 & 0.00 & 0.00 & 0.00 & 0.00 \\
        $S_2$     & 0.20 & -0.90 & 0.56 & 0.00 & 0.00 & 0.00 & 0.14 & 0.00 & 0.00 & 0.00 & 0.00 & 0.00 \\
        $S_3$     & 0.00 & 0.20 & -0.90 & 0.56 & 0.00 & 0.00 & 0.00 & 0.14 & 0.00 & 0.00 & 0.00 & 0.00 \\
        $S_4$     & 0.00 & 0.00 & 0.20 & -0.34 & 0.00 & 0.00 & 0.00 & 0.00 & 0.14 & 0.00 & 0.00 & 0.00 \\
        $S_5$     & 0.20 & 0.00 & 0.00 & 0.00 & -0.90 & 0.14 & 0.56 & 0.00 & 0.00 & 0.00 & 0.00 & 0.00 \\
        $S_6$     & 0.00 & 0.00 & 0.00 & 0.00 & 0.20 & -0.76 & 0.00 & 0.00 & 0.00 & 0.56 & 0.00 & 0.00 \\
        $S_7$     & 0.00 & 0.20 & 0.00 & 0.00 & 0.20 & 0.00 & -1.10 & 0.56 & 0.00 & 0.14 & 0.00 & 0.00 \\
        $S_8$     & 0.00 & 0.00 & 0.20 & 0.00 & 0.00 & 0.00 & 0.20 & -1.10 & 0.56 & 0.00 & 0.14 & 0.00 \\
        $S_9$     & 0.00 & 0.00 & 0.00 & 0.20 & 0.00 & 0.00 & 0.00 & 0.20 & -0.54 & 0.00 & 0.00 & 0.14 \\
        $S_{10}$  & 0.00 & 0.00 & 0.00 & 0.00 & 0.00 & 0.20 & 0.20 & 0.00 & 0.00 & -0.96 & 0.56 & 0.00 \\
        $S_{11}$  & 0.00 & 0.00 & 0.00 & 0.00 & 0.00 & 0.00 & 0.00 & 0.20 & 0.00 & 0.20 & -0.96 & 0.56 \\
        $S_{12}$  & 0.00 & 0.00 & 0.00 & 0.00 & 0.00 & 0.00 & 0.00 & 0.00 & 0.20 & 0.00 & 0.20 & -0.40
    \end{tabular}
    \caption{Матрица интенсивностей переходов}
    \label{tab:intensity_matrix}
\end{table}

\subsection*{Характеристика системы 1}

\begin{table}[H]
    \centering
    \begin{tabular}{|l|l|r|l|}
    \hline
    Характеристика & Прибор & Значение & Формула расчета \\
    \hline
    \multirow{3}{*}{Нагрузка} 
    & П1 & 2.8000 & $\rho_1 = \frac{\lambda p_1}{\mu} = \frac{0.7 \cdot 0.8}{5} \cdot 20 = 2.8$ \\
    & П2 & 0.7000 & $\rho_2 = \frac{\lambda p_2}{\mu} = \frac{0.7 \cdot 0.2}{5} \cdot 20 = 0.7$ \\
    & Сумм & 3.5000 & $\rho = \rho_1 + \rho_2$ \\
    \hline
    \multirow{3}{*}{Загрузка} 
    & П1 & 0.9702 & $Y_1 = 1 - P_1 - P_5 - P_6$ \\
    & П2 & 0.5434 & $Y_2 = 1 - P_1 - P_2 - P_3 - P_4$ \\
    & Сред & 0.7568 & $Y_{ср} = \frac{Y_1 + Y_2}{2}$ \\
    \hline
    \multirow{3}{*}{Длина очереди} 
    & П1 & 1.5404 & $L_1 = (P_3 + P_8 + P_{11}) + 2(P_4 + P_9 + P_{12})$ \\
    & П2 & 0.2237 & $L_2 = P_6 + P_{10} + P_{11} + P_{12}$ \\
    & Сумм & 1.7641 & $L = L_1 + L_2$ \\
    \hline
    \multirow{3}{*}{Число заявок} 
    & П1 & 2.5106 & $M_1 = L_1 + Y_1$ \\
    & П2 & 0.7671 & $M_2 = L_2 + Y_2$ \\
    & Сумм & 3.2777 & $M = M_1 + M_2$ \\
    \hline
    \multirow{3}{*}{Время ожидания} 
    & П1 & 2.7507 & $W_1 = \frac{L_1}{\lambda p_1}$ \\
    & П2 & 1.5981 & $W_2 = \frac{L_2}{\lambda p_2}$ \\
    & Сумм & 2.5202 & $W = \frac{L}{\lambda}$ \\
    \hline
    \multirow{3}{*}{Время пребывания} 
    & П1 & 7.7507 & $U_1 = W_1 + \frac{1}{\mu}$ \\
    & П2 & 6.5981 & $U_2 = W_2 + \frac{1}{\mu}$ \\
    & Сумм & 14.3488 & $U = U_1 + U_2$ \\
    \hline
    \multirow{3}{*}{Вероятность потери} 
    & П1 & 0.3660 & $e_{1} = (P_4 + P_{9} + P_{12}) \cdot p_1$ \\
    & П2 & 0.0313 & $e_{2} = (P_{6} + P_{10} + P_{11} + P_{12}) \cdot p_2$ \\
    & Сумм & 0.3973 & $e_{} = e_{1} + e_{2}$ \\
    \hline
    \multirow{3}{*}{Производительность} 
    & П1 & 0.3551 & $A_1 = \lambda p_1(1 - e_{1})$ \\
    & П2 & 0.1356 & $A_2 = \lambda p_2(1 - e_{2})$ \\
    & Сумм & 0.4907 & $A = A_1 + A_2$ \\
    \hline
    \end{tabular}
    \caption{Характеристики системы массового обслуживания с формулами расчета}
    \label{tab:queueing_system}
\end{table}

\section*{Вывод}


\end{document}
\includegraphics[width=.9\textwidth]{123}



